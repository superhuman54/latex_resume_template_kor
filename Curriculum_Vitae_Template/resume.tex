%-------------------------
% Resume in Latex
% Original Author : Sourabh Bajaj
% Adaptation : Hyunggi Chang
% License : MIT
%------------------------

\documentclass[letterpaper,11pt]{article}

\usepackage{latexsym}
\usepackage[empty]{fullpage}
\usepackage{titlesec}
\usepackage{marvosym}
\usepackage[usenames,dvipsnames]{color}
\usepackage{verbatim}
\usepackage{enumitem}
\usepackage[hidelinks]{hyperref}
\usepackage{fancyhdr}
\usepackage[english]{babel}
\usepackage{tabularx}
\usepackage{amsmath}
\usepackage{kotex} % Enable Korean!

\pagestyle{fancy}
\fancyhf{} % clear all header and footer fields
\fancyfoot{}
\renewcommand{\headrulewidth}{0pt}
\renewcommand{\footrulewidth}{0pt}

% Adjust margins
\addtolength{\oddsidemargin}{-0.5in}
\addtolength{\evensidemargin}{-0.5in}
\addtolength{\textwidth}{1in}
\addtolength{\topmargin}{-0.5in}
\addtolength{\textheight}{1.0in}

\urlstyle{same}

\raggedbottom
\raggedright
\setlength{\tabcolsep}{0in}

% Sections formatting
\titleformat{\section}{
  \vspace{-4pt}\scshape\raggedright\large
}{}{0em}{}[\color{black}\titlerule \vspace{-2pt}]

%-------------------------
% Custom commands
\newcommand{\resumeItem}[1]{
  \item\small{
    {#1 \vspace{-2pt}}
  }
}

\newcommand{\resumeSummary}[1]{
  \item
    \begin{tabular*}{0.97\textwidth}[t]{l@{\extracolsep{\fill}}r}
      #1
    \end{tabular*}
}

\newcommand{\resumeSubheading}[4]{
  \vspace{-1pt}\item
    \begin{tabular*}{0.97\textwidth}[t]{l@{\extracolsep{\fill}}r}
      \textbf{#1} & #2 \\
      \textit{\small#3} & \textit{\small #4} \\
    \end{tabular*}\vspace{-5pt}
}

\newcommand{\resumeEmployment}[4]{
  \vspace{-1pt}\item
    \begin{tabular*}{0.97\textwidth}[t]{l@{\extracolsep{\fill}}r}
      \textbf{#1} & #2 \\
      \textit{\small#3} & \textit{\small #4} \\
    \end{tabular*}\vspace{-5pt}
}

\newcommand{\resumeProject}[2]{
  \vspace{-1pt}\item
    \begin{tabular*}{0.97\textwidth}[t]{l@{\extracolsep{\fill}}r}
      \textbf{#1} \\
      \small{#2} \\
    \end{tabular*}\vspace{-5pt}
}

\newcommand{\resumeResearch}[5]{
  \vspace{-1pt}\item
    \begin{tabular*}{0.97\textwidth}[t]{l@{\extracolsep{\fill}}r}
      \textbf{#1} & #2 \\
      \textit{\small#3} {\small #4 \vspace{-2pt}} & \textit{\small #5} \\
    \end{tabular*}\vspace{-5pt}
}

\newcommand{\resumeTalk}[2]{
  \vspace{-1pt}\item
    \begin{tabular*}{0.97\textwidth}[t]{l@{\extracolsep{\fill}}r}
      \textbf{#1} & #2 \\
    \end{tabular*}\vspace{-5pt}
}

\newcommand{\resumeSkills}[1]{
  \item
    \begin{tabular*}{0.97\textwidth}[t]{l@{\extracolsep{\fill}}r}
      #1
    \end{tabular*}
}

\newcommand{\resumeCommunity}[3]{
  \vspace{-1pt}\item
    \begin{tabular*}{0.97\textwidth}[t]{l@{\extracolsep{\fill}}r}
      \textbf{#1} & #2 \\
      \textit{\small#3} \\
    \end{tabular*}\vspace{-5pt}
}

\newcommand{\resumeSubItem}[2]{\resumeItem{#1}{#2}\vspace{-4pt}}

\renewcommand{\labelitemii}{$\circ$}

\newcommand{\resumeSubHeadingListStart}{\begin{itemize}[leftmargin=*]}
\newcommand{\resumeSubHeadingListEnd}{\end{itemize}}

\newcommand{\resumeEmploymentListStart}{\begin{itemize}[leftmargin=*]}
\newcommand{\resumeEmploymentListEnd}{\end{itemize}}

\newcommand{\resumeItemListStart}{\begin{itemize}}
\newcommand{\resumeItemListEnd}{\end{itemize}\vspace{-5pt}}

%-------------------------------------------
%%%%%%  CV STARTS HERE  %%%%%%%%%%%%%%%%%%%%%%%%%%%%


\begin{document}

%----------HEADING-----------------
\begin{tabular*}{\textwidth}{l@{\extracolsep{\fill}}r}
  \textbf{\href{https://github.com/superhuman54}{\Large 김기환}} & Github: \href{https://github.com/superhuman54}{superhuman54} \\
  \href{mailto:superhuman54.tech@gmail.com}{superhuman54.tech@gmail.com} & Mobile : 010-4691-0485 \\
  데이터 엔지니어 & 서울 거주
\end{tabular*}

%-----------Summary-----------------
\section{Summary}
    \resumeSummary{
    AWS 클라우드 기반 대용량 데이터 플랫폼 운영 및 ETL 개발 전문가입니다. \\ 
    하루 2억 5천만 레코드를 처리하는 파이프라인 설계와 운영 경험을 보유하고 있습니다. \\ 
    데이터 엔지니어링과 백엔드 개발 분야에서 9년 이상의 실무 경험을 가지고 있습니다.
    }

%-----------EXPERIENCE-----------------
\section{Employment}
  \resumeEmploymentListStart
    \resumeEmployment
      {드림어스컴퍼니}{서울, 대한민국}
      {데이터 엔지니어 (정규직)}{2020.02 - 현재}
        \resumeItemListStart
            \resumeItem{AWS 클라우드 데이터 플랫폼 운영 및 아키텍처 설계}
            \resumeItem{Spark, Polars 기반 대용량 ETL 파이프라인 개발 및 최적화}
            \resumeItem{Python 기반 API 개발 및 운영으로 \href{https://www.music-flo.com}{\color{blue} FLO 음악 스트리밍 서비스} 지원}
            \resumeItem{하루 2억 5천만 레코드 처리하는 사용자 로그 파이프라인 설계}
            \resumeItem{침해곡 탐지 파이프라인 개선으로 하루 최대 16만곡 검수 자동화}
            \resumeItem{Trino 쿼리 엔진 운영 및 성능 최적화}
        \resumeItemListEnd   
        
    \resumeEmployment
      {테이블링}{서울, 대한민국}
      {서버 개발자 (정규직)}{2017.09 - 2019.07}
      \resumeItemListStart
          \resumeItem{AWS 클라우드 기반 서버 인프라 운영 및 관리}
          \resumeItem{Node.js 기반 API 개발 및 운영}
          \resumeItem{레스토랑 검색 서비스 백엔드 개발}
      \resumeItemListEnd

    \resumeEmployment
      {네이버}{서울, 대한민국}
      {안드로이드 개발자 (정규직)}{2017.06 - 2017.09}
      \resumeItemListStart
          \resumeItem{네이버 TV 안드로이드 앱 동영상 댓글 기능 개발}
          \resumeItem{실시간 단체 채팅 안드로이드 클라이언트 개발}
      \resumeItemListEnd

    \resumeEmployment
      {키위플러스}{서울, 대한민국}
      {풀스택 개발자 (정규직)}{2014.11 - 2016.08}
      \resumeItemListStart
          \resumeItem{Spring 기반 REST API 서버 개발}
          \resumeItem{스마트워치 안드로이드 SDK 및 시스템 앱 개발}
      \resumeItemListEnd
      
  \resumeEmploymentListEnd

%-----------Projects-----------------

\section{Projects}
  \resumeSubHeadingListStart
    \resumeProject
      {사용자 로그 파이프라인}
         {하루 약 2억 5천만 레코드를 처리하는 대용량 ETL 파이프라인 설계 및 운영}
         \resumeItemListStart
             \resumeItem{대용량 데이터 처리 아키텍처 설계 \\
             \textbf{주요 성과}: \\ 
             1. Spark 최적화를 통한 처리 성능 40\% 향상 \\
             2. 데이터 파이프라인 안정성 99.9\% 달성 \\
             3. 실시간 모니터링 시스템 구축}
             \resumeItem{AWS EMR, Airflow, Kubernetes 기반 워크플로우 구축}
             \resumeItem{데이터 품질 관리 및 모니터링 시스템 개발}
         \resumeItemListEnd
    
    \resumeProject
      {침해곡 탐지 파이프라인}
      {하루 최대 16만곡을 자동 검수하는 지적재산권 침해 탐지 시스템}
      \resumeItemListStart
        \resumeItem{음원 메타데이터 분석 및 유사도 검증 알고리즘 개발}
        \resumeItem{머신러닝 기반 침해곡 자동 탐지 시스템 구축}
        \resumeItem{파이프라인 성능 최적화로 처리 시간 60\% 단축}
    \resumeItemListEnd
    
    \resumeProject
      {쿼리 엔진 Trino 운영}
    {DW 조회를 위한 분산 쿼리 엔진 운영 및 최적화}
    \resumeItemListStart
        \resumeItem{Trino 클러스터 구성 및 성능 튜닝}
        \resumeItem{OOM 문제 해결 및 느린 쿼리 분석}
        \resumeItem{메모리 최적화를 통한 쿼리 성능 개선}
    \resumeItemListEnd

  \resumeSubHeadingListEnd

%-----------Skills-----------------

\section{Skills}
  \resumeSubHeadingListStart
    \resumeSkills{\textbf{Programming Languages} - Python, Java 8, Scala, Shell Script, SQL}
    \resumeSkills{\textbf{Big Data \& Analytics} - Spark 3, Hadoop 3, Hive 3, Trino 435, Polars}
    \resumeSkills{\textbf{Cloud Platforms} - AWS (EMR, MWAA, EKS, S3, EC2, RDS)}
    \resumeSkills{\textbf{Data Pipeline \& Workflow} - Apache Airflow 2, Kafka, ETL/ELT}
    \resumeSkills{\textbf{Container \& Orchestration} - Kubernetes 1.3, Docker}
    \resumeSkills{\textbf{Web Frameworks} - SpringBoot 2, FastAPI, Node.js}
    \resumeSkills{\textbf{Databases} - PostgreSQL, MySQL, DynamoDB}
  \resumeSubHeadingListEnd

%-----------EDUCATION-----------------
\section{Education}
  \resumeSubHeadingListStart
    \resumeSubheading
      {영남대학교}{대구, 대한민국}
      {전기공학과 학사}{2005.03 - 2013.02}

  \resumeSubHeadingListEnd

%-----------Certifications-----------------
\section{Certifications}
    \resumeSubHeadingListStart
        \resumeItem{\textbf{RHCSA (Red Hat Certified System Administrator)} - 2019.09}
        \resumeItem{\textbf{CCNA (Cisco Certified Network Associate Routing and Switching)} - 2019.11}
    \resumeSubHeadingListEnd

%-----------Publications-----------------
\section{Publications}
    \resumeSubHeadingListStart
        \resumeItem{\textbf{안드로이드 데이터베이스} - 아담 스트라우드 저, 김기환 공역, 2017.03}
    \resumeSubHeadingListEnd

%-----------Tech Articles-----------------
\section{Technical Articles \& Troubleshooting}
    \resumeSubHeadingListStart
        \resumeItem{Airflow가 생성한 K8s Pod 회수 실패 해결}
        \resumeItem{2개의 Airflow와 1개의 K8s 운영 최적화}
        \resumeItem{Spark SaveAsTable() 동작 원리 분석}
        \resumeItem{Spark → DynamoDB 처리량 저하 문제 해결}
        \resumeItem{Amazon EMR 6.12와 AWS Java SDK 충돌 해결}
        \resumeItem{Trino ORC vs Parquet 성능 비교 분석}
    \resumeSubHeadingListEnd

%-------------------------------------------
\end{document}
