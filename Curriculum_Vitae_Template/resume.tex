%-------------------------
% Resume in Latex
% Original Author : Sourabh Bajaj
% Adaptation : Hyunggi Chang
% License : MIT
%------------------------

\documentclass[letterpaper,11pt]{article}

\usepackage{latexsym}
\usepackage[empty]{fullpage}
\usepackage{titlesec}
\usepackage{marvosym}
\usepackage[usenames,dvipsnames]{color}
\usepackage{verbatim}
\usepackage{enumitem}
\usepackage[hidelinks]{hyperref}
\usepackage{fancyhdr}
\usepackage[english]{babel}
\usepackage{tabularx}
\usepackage{amsmath}
\usepackage{kotex} % Enable Korean!

\pagestyle{fancy}
\fancyhf{} % clear all header and footer fields
\fancyfoot{}
\renewcommand{\headrulewidth}{0pt}
\renewcommand{\footrulewidth}{0pt}

% Adjust margins
\addtolength{\oddsidemargin}{-0.5in}
\addtolength{\evensidemargin}{-0.5in}
\addtolength{\textwidth}{1in}
\addtolength{\topmargin}{-0.5in}
\addtolength{\textheight}{1.0in}

\urlstyle{same}

\raggedbottom
\raggedright
\setlength{\tabcolsep}{0in}

% Sections formatting
\titleformat{\section}{
  \vspace{-4pt}\scshape\raggedright\large
}{}{0em}{}[\color{black}\titlerule \vspace{-2pt}]

%-------------------------
% Custom commands
\newcommand{\resumeItem}[1]{
  \item\small{
    {#1 \vspace{-2pt}}
  }
}

\newcommand{\resumeSummary}[1]{
  \item
    \begin{tabular*}{0.97\textwidth}[t]{l@{\extracolsep{\fill}}r}
      #1
    \end{tabular*}
}

\newcommand{\resumeSubheading}[4]{
  \vspace{-1pt}\item
    \begin{tabular*}{0.97\textwidth}[t]{l@{\extracolsep{\fill}}r}
      \textbf{#1} & #2 \\
      \textit{\small#3} & \textit{\small #4} \\
    \end{tabular*}\vspace{-5pt}
}

\newcommand{\resumeEmployment}[4]{
  \vspace{-1pt}\item
    \begin{tabular*}{0.97\textwidth}[t]{l@{\extracolsep{\fill}}r}
      \textbf{#1} & #2 \\
      \textit{\small#3} & \textit{\small #4} \\
    \end{tabular*}\vspace{-5pt}
}

\newcommand{\resumeProject}[2]{
  \vspace{-1pt}\item
    \begin{tabular*}{0.97\textwidth}[t]{l@{\extracolsep{\fill}}r}
      \textbf{#1} \\
      \small{#2} \\
    \end{tabular*}\vspace{-5pt}
}

\newcommand{\resumeResearch}[5]{
  \vspace{-1pt}\item
    \begin{tabular*}{0.97\textwidth}[t]{l@{\extracolsep{\fill}}r}
      \textbf{#1} & #2 \\
      \textit{\small#3} {\small #4 \vspace{-2pt}} & \textit{\small #5} \\
    \end{tabular*}\vspace{-5pt}
}

\newcommand{\resumeTalk}[2]{
  \vspace{-1pt}\item
    \begin{tabular*}{0.97\textwidth}[t]{l@{\extracolsep{\fill}}r}
      \textbf{#1} & #2 \\
    \end{tabular*}\vspace{-5pt}
}

\newcommand{\resumeSkills}[1]{
  \item
    \begin{tabular*}{0.97\textwidth}[t]{l@{\extracolsep{\fill}}r}
      #1
    \end{tabular*}
}

\newcommand{\resumeCommunity}[3]{
  \vspace{-1pt}\item
    \begin{tabular*}{0.97\textwidth}[t]{l@{\extracolsep{\fill}}r}
      \textbf{#1} & #2 \\
      \textit{\small#3} \\
    \end{tabular*}\vspace{-5pt}
}

\newcommand{\resumeSubItem}[2]{\resumeItem{#1}{#2}\vspace{-4pt}}

\renewcommand{\labelitemii}{$\circ$}

\newcommand{\resumeSubHeadingListStart}{\begin{itemize}[leftmargin=*]}
\newcommand{\resumeSubHeadingListEnd}{\end{itemize}}

\newcommand{\resumeEmploymentListStart}{\begin{itemize}[leftmargin=*]}
\newcommand{\resumeEmploymentListEnd}{\end{itemize}}

\newcommand{\resumeItemListStart}{\begin{itemize}}
\newcommand{\resumeItemListEnd}{\end{itemize}\vspace{-5pt}}

%-------------------------------------------
%%%%%%  CV STARTS HERE  %%%%%%%%%%%%%%%%%%%%%%%%%%%%


\begin{document}

%----------HEADING-----------------
\begin{tabular*}{\textwidth}{l@{\extracolsep{\fill}}r}
  \textbf{\href{https://superhuman54.github.io}{\Large 김기환}} & Github: \href{https://github.com/superhuman54}{superhuman54} \\
  \href{https://superhuman54.github.io}{superhuman54.github.io} & Email : \href{mailto:superhuman54.tech@gmail.com}{superhuman54.tech@gmail.com} \\
  데이터 엔지니어 & Mobile : 010-4691-0485
\end{tabular*}
%-----------Summary-----------------
\section{}
  \resumeSubHeadingListStart
    \resumeSummary{
    \textbf{매일 2.5억} 건 이상의 대규모 로그 데이터를 안정적으로 처리하며, 데이터 문제의 근본 원인을 찾아 해결하는 \\ 
    데이터 엔지니어입니다. 단순한 파이프라인 구축을 넘어, 코드 레벨의 깊이 있는 분석과 \textbf{집요한 트러블슈팅}으로 \\ 
    시스템의 한계를 극복합니다. 안정성과 효율성을 모두 갖춘 데이터 인프라를 통해 조직의 운영 비용을 절감하고, \\ 
    데이터의 가치를 극대화합니다.
    }
  \resumeSubHeadingListEnd

%-----------EXPERIENCE-----------------
\section{경력 사항}
  \resumeEmploymentListStart
    \resumeEmployment
      {드림어스컴퍼니, www.music-flo.com}{서울, 대한민국}
      {데이터 엔지니어 (정규직)}{2020.02 - 현재}
        \resumeItemListStart
            \resumeItem{AWS 클라우드 기반 대용량 데이터 파이프라인 구축 및 운영}
            \resumeItem{Spark, Polars 데이터 처리 및 분석 시스템 개발}
            \resumeItem{사내 데이터 플랫폼 Trino 운영 및 최적화}
            \resumeItem{Amazon S3, Amazon Glue Catalog 기반 데이터레이크 및 데이터웨어하우스 운영}
            \resumeItem{Amazon Glue ETL 및 Spark 활용 준실시간 데이터 파이프라인 구축 및 운영}
            \resumeItem{Amazon MWAA 활용 데이터 파이프라인 스케줄링}
            \resumeItem{음원 침해곡 탐지 파이프라인 운영}
            \resumeItem{FastAPI 기반 오프라인 추천 시스템 REST API 서버 개발}
        \resumeItemListEnd   
        
    \resumeEmployment
      {테이블링, www.tabling.co.kr}{서울, 대한민국}
      {서버 개발자 (정규직)}{2017.09 - 2019.07}
      \resumeItemListStart
          \resumeItem{AWS 클라우드 기반 서버 인프라 운영 및 관리}
          \resumeItem{API 백엔드 개발}
      \resumeItemListEnd

    \resumeEmployment
      {네이버, https://tv.naver.com}{성남, 대한민국}
      {안드로이드 개발자 (정규직)}{2017.06 - 2017.09}
      \resumeItemListStart
          \resumeItem{네이버 TV 안드로이드 앱 동영상 댓글 기능 개발}
          \resumeItem{실시간 단체 채팅 안드로이드 클라이언트 개발}
      \resumeItemListEnd

    \resumeEmployment
      {키위플러스, www.kiwiplus.co.kr}{서울, 대한민국}
      {풀스택 개발자 (정규직)}{2014.11 - 2016.08}
      \resumeItemListStart
          \resumeItem{SpringBoot 기반 REST API 서버 개발}
          \resumeItem{스마트워치 안드로이드 SDK 및 시스템 앱 개발}
      \resumeItemListEnd

    \resumeEmployment
      {빅맨게임즈}{판교, 대한민국}
      {클라이언트, 서버 개발자 (정규직)}{2014.05 - 2014.07}
      \resumeItemListStart
          \resumeItem{Spring framework 기반 REST API 서버 개발}
          \resumeItem{게임 클라이언트 개발}
      \resumeItemListEnd
      
  \resumeEmploymentListEnd

%-----------Projects-----------------

\section{프로젝트}
  \resumeSubHeadingListStart
    \resumeProject
      {Spark History MCP + AI Agent로 Spark 분석 자동화}
      {Spark 작업 실패 시 자동으로 원인 분석 및 해결책을 제시하는 AI 기반 자동 분석 시스템 구축}
      \resumeItemListStart
        \resumeItem{SparkListener 구현 및 EMR 통합으로 Spark 이벤트 실시간 감지}
        \resumeItem{n8n AI Agent 워크플로우 구축으로 자동 분석 결과 Slack 전송}
        \resumeItem{\textbf{사람의 수동 분석 시간 100\% 감소}}
      \resumeItemListEnd
    
    \resumeProject
      {액션로그 파이프라인}
         {하루 약 2억 5천만 레코드를 처리하는 대용량 ETL 파이프라인 설계 및 운영}
         \resumeItemListStart
            \resumeItem{Kafka Connect 기반 스트리밍 데이터 수집 및 Amazon Glue ETL로 1시간 주기 준실시간 파티션 등록}
            \resumeItem{메달리온 아키텍처 설계 및 클라이언트 조회 패턴 분석 기반 파티셔닝/버켓팅 최적화, \textbf{조회 속도 2.5배 증가}}
         \resumeItemListEnd
    
    \resumeProject
      {침해곡 탐지 파이프라인}
      {음원 메타데이터 분석 및 유사도 검증 알고리즘을 이용하여 하루 최대 16만곡을 검수하는 지적재산권 침해 탐지 시스템}
      \resumeItemListStart
        \resumeItem{벡터 유사도와 텍스트 유사도 기반 침해곡 탐지 시스템 구축}
        \resumeItem{Polars와 Kubernetes 조합으로 \textbf{하루 운영 비용 73\% 절감}, \textbf{총 소요시간 53\% 단축} (14시간→6.5시간)}
        \resumeItem{Polars와 벡터 데이터베이스(PgVector)를 적용하여 \textbf{동/변조곡 탐지 정확도 18\%p}(5\% $\rightarrow$ 23\%) 향상}
    \resumeItemListEnd
    
    \resumeProject
      {쿼리 엔진 Trino 운영}
    {DW 조회를 위한 분산 쿼리 엔진 운영}
    \resumeItemListStart
        \resumeItem{Prometheus와 Grafana 기반 모니터링 시스템 구축으로 Observability 확보로 인해 \textbf{불필요한 Worker EC2 비용 월 45\% 절감}}
        \resumeItem{Presto에서 Trino로의 전환 후 \textbf{쿼리 처리 속도 15\% 증가}, 전환 과정에서 발생한 다양한 운영 이슈(OOM, 느린 쿼리)를 체계적으로 해결하여, 멀티테넌트 환경에서의 안정적인 서비스 제공}

    \resumeItemListEnd

    \resumeProject
      {Amazon S3, Amazon Glue Catalog 기반 데이터레이크, 데이터웨어하우스 운영}
    {FLO 서비스의 데이터 인프라 운영 및 자동화}
    \resumeItemListStart
        \resumeItem{Amazon S3 기반 데이터레이크 및 데이터웨어하우스 운영}
        \resumeItem{Jenkins에서 Airflow로 스케줄러 이관}
        \resumeItem{Spark on EMR을 활용한 데이터마트 구성}
        \resumeItem{AWS EMR, Amazon MWAA, Amazon EKS를 활용한 워크플로우 구축}
    \resumeItemListEnd

    \resumeProject
      {PySpark에서 Polars로의 마이그레이션}
      {소규모 EMR 클러스터에서 단일 EC2 인스턴스 기반 Polars로 전환하여 성능과 비용을 동시에 최적화}
      \resumeItemListStart
        \resumeItem{집계 작업 수행 시간 \textbf{85.42\% 감소} (4분 5초 → 36초), 전체 소요 시간 10분 → 1분}
        \resumeItem{하드웨어 스펙 축소: 2노드(8 코어) → 단일 노드(4 코어), 메모리 48GB → 12GB}
        \resumeItem{일일 운영 비용 \textbf{71.13\% 절감} (\$19.81 → \$5.72)}
      \resumeItemListEnd

    \resumeProject
      {추천 데이터 HTTP 서버 구축}
    {FastAPI 기반 추천 데이터 HTTP 서버 개발}
    \resumeItemListStart
        \resumeItem{기존 Flask 대비 비동기 처리 기반 FastAPI 도입으로 API 요청 \textbf{지연시간 76\% 감축} 및 동시성 향상}
        \resumeItem{타입 힌팅 적용과 Dependency Injection으로 코드 안정성과 유지보수성 강화}
    \resumeItemListEnd

  \resumeSubHeadingListEnd

%-----------Skills-----------------

\section{기술 스택}
  \resumeSubHeadingListStart
    \resumeSkills{\textbf{Programming Languages}}
    \begin{itemize}[leftmargin=1em]
        \item Python (★★★★☆)
        \item Java 8 (★★★★☆)
        \item Shell Script (★★★☆☆)
        \item SQL (★★☆☆☆)
    \end{itemize}
    \resumeSkills{\textbf{Big Data \& Analytics}}
    \begin{itemize}[leftmargin=1em]
        \item Spark 3 (★★★★☆)
        \item Hadoop 3 (★★★☆☆)
        \item Hive 3 (★★☆☆☆)
        \item Trino 435 (★★★★☆)
        \item Polars (★★★☆☆)
    \end{itemize}
    \resumeSkills{\textbf{Cloud Platforms} - AWS (EMR, Glue, MWAA, EKS, S3, EC2, DynamoDB, IAM, CloudWatch, Lambda)}
    \resumeSkills{\textbf{Data Pipeline \& Workflow} - Apache Airflow 2, Kafka S3 Sink Connect}
    \resumeSkills{\textbf{Container \& Orchestration}}
    \begin{itemize}[leftmargin=1em]
        \item Kubernetes 1.3: Pod, Deployment 개발
        \item Docker
    \end{itemize}
    \resumeSkills{\textbf{Web Frameworks} - SpringBoot 2, FastAPI}
  \resumeSubHeadingListEnd

%-----------EDUCATION-----------------
\section{학력}
  \resumeSubHeadingListStart
    \resumeSubheading
      {영남대학교}{대구, 대한민국}
      {전기공학과 학사}{2005 - 2013}

  \resumeSubHeadingListEnd

%-----------Certifications-----------------
\section{자격증}
    \resumeSubHeadingListStart
        \resumeItem{\textbf{RHCSA (Red Hat Certified System Administrator)} - 2019.09}
        \resumeItem{\textbf{CCNA (Cisco Certified Network Associate Routing and Switching)} - 2019.11}
    \resumeSubHeadingListEnd

\end{document}
